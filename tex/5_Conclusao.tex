\label{chap:conclusao}

Nesse trabalho, foi apresentado um método de MtA baseado no \textit{framework} MetaStream. Este método aprimorou o MetaStream original pela adição de meta-atributos mais modernos e informativos e também pela inclusão do aprendizado incremental no nível MtA por meio do LightGBM como meta-classificador. Embora ambas as estratégias tenham desempenhado de maneira similar, a estratégia incremental apresenta maior robustez devido seu menor tempo para indução do modelo e menor uso de memória. Os resultados experimentais apresentaram que o meta-classificador pode consistentemente recomendar o melhor algorítimo para uma dada janela no fluxo de dados, conduzindo a um ganho acumulado de performance ao longo do tempo.

Como sugestões para trabalhos futuros, é possível estender o conjunto de algoritmos nível base para mais de duas classes, possibilitando a inclusão de mais vieses a serem selecionados para o nível base. Outra possibilidade é utilizar classificadores incrementais no nível base, melhorando ainda mais a performance desse framework. A extração de meta-atributos de forma incremental também é possível, incorporada ao MetaStream, permitiria a inclusão de algoritmo que no modo tradicional, em lote, seriam custoso demais para serem adicionados como já foi discutido. Alternativamente, embora tenham sido utilizados meta-atributos tradicionais, meta-atributos de séries temporais podem apresentar um grande poder discriminativo para esse tipo de problema, melhorando a performance meta do sistema. Por fim, uma abordagem recente é regredir modelos baseado em suas acurácias, como forma de recomendação, permitindo treinar de forma irrestrita classificadores a nível base sem as limitações de insuficiência estatística.

% Como trabalhos futuros, gostaria de estender a classificação a mais de duas classes, isto é, uma classificação multi-classe, tornando possível selecionar entre múltiplos espaços de hipótese no nível base. Outra ideia que viável é usar de algoritmos incrementais para os meta-atributos, tornando possível diminuir o tempo de processamento, ou adicionar meta-atributos mais custosos porém mais informativos. Também, como proposto pelo artigo original \cite{rossi2014metastream}, atributos de séries temporais podem apresentar um grande poder discriminativo para esse tipo de problema. Uma abordagem recente é regredir modelos baseado em suas acurácias como forma de recomendação, o qual também a aplicável ao problema aqui proposto e permite estender a outros conjuntos de dados