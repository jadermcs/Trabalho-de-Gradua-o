Muitas companhias vêm tirando proveito de mineração de dados para identificar informações valiosas em conjuntos de dados massivos gerados em alta frequência, o chamado \textit{Big Data}. Técnicas de Aprendizado de Máquina podem ser aplicadas para descoberta de informação, visto que podem extrair padrões dos dados para induzir modelos que preverão eventos futuros. Entretanto, ambientes dinâmicos e progressivos comumente geram fluxos de dados não estacionários. Logo, modelos treinados nesse cenário costumam perecer com o tempo pela sazonalidade ou mudança de conceito. O retreinamento periódico pode ajudar, mas um espaço de hipóteses fixo pode não ser o mais apropriado ao fenômeno. Uma solução alternativa é usar meta-aprendizado para uma contínua seleção de algoritmos em ambientes que mudam com o tempo, escolhendo o viés que melhor condiz com os dados atuais. Nesse trabalho, apresentamos um \textit{framework}\footnote{Modelo de aplicação em software.} aprimorado para seleção de algoritmos em fluxos de dados baseado no MetaStream. Nossa abordagem usa meta-aprendizado e aprendizado incremental para ativamente selecionar o melhor algoritmo para o presente conceito em ambientes que mudam com o tempo. Ao contrário de trabalhos prévios, nós usamos uma coleção diversificada de meta-atributos estado-da-arte e uma abordagem de aprendizado incremental para o nível meta baseada no algoritmo LightGBM. Os resultados mostram que essa nova estratégia pode aprimorar a acurácia de recomendação do melhor algoritmo em dados que mudam com o tempo.