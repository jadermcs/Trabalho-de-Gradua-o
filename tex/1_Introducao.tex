Sistemas digitais estão presentes no dia-a-dia das pessoas nas mais diversas formas, \textit{smartphones} com seus diversos aplicativos, \textit{smart things} (TVs,
relógios, etc.), sensores, computadores de bordo e muitos outros dispositivos exemplificam um cenário que, em função do progresso tecnológico, apresentará um crescimento ainda maior para os próximos anos \cite{owidtechnologyadoption,atzori2010internet}. Tais sistemas, por estarem conectados à redes
internas e externas, têm a capacidade de consumir e gerar um grande volume de dados. Esse fenômeno de geração e armazenamento de uma quantidade massiva de
dados recebeu um nome, ``\textit{Big Data}''~\cite{mcafee2012big, dean2008mapreduce}.

Essa grande quantidade de informação, mesmo sem qualquer filtragem ou supervisão, possibilitou tarefas de grande valor comercial, como recomendação de conteúdo e processamento de linguagem natural \cite{halevy2009unreasonable}, se tornado então, um dos bens mais valiosos do mundo moderno \cite{economist2017world}. Companhias como Google\regsymbol{}, Facebook\regsymbol{}, Netflix\regsymbol{} e Amazon\regsymbol{} estão entre as empresas que competem por esse bem digital com o objetivo de extrair informações valiosas aos seus usuários e ao mercado como um todo \cite{tarnoff2018big, finger2014data}.

Fluxos de dados, ou \textit{Data Streams}, são ambientes típicos em infraestruturas de \textit{Big Data}, ao contrário de dados em lote, que tem inicio, fim e tamanho bem definidos, um fluxo de dados é uma fonte contínua (interminável) de dados, exemplos são redes de sensores e monitoramento, registros de sistemas web, transações em comércio eletrônico, entre outros. Do ponto de vista de engenharia, pela natureza dos dados, há imposições limitantes quanto a leitura e manipulação, o que torna os processos de mineração para esses
ambientes uma tarefa não trivial \cite{gama2007learning}.

Outro aspecto limitado para esses sistemas são as técnicas de mineração de dados \cite{gama2007learning}, dentre elas em especial o Aprendizado de Máquina (AM), campo que estuda métodos para a extração de padrões dos dados, para que esse conhecimento seja aplicado em tarefas futuras \cite{mitchell1997machine,friedman2001elements}. Como todo método indutivo, algoritmos tradicionais de AM assumem duas premissas que devem ser satisfeitas para sua validade: ($i$) o futuro deve se comportar como o passado e ($ii$) eventos futuros devem ser independentes de eventos passado \cite{vapnik2013nature}. Entretanto, em um fluxo de dados, a distribuição que gera os dados usualmente está mudando com o tempo e dependências temporais podem ocorrer.  Logo, essas premissas podem não se manter e o desempenho dos modelos induzidos podem perecer com o passar do tempo, causando uma má experiência aos usuários desses sistemas \cite{gama2007learning, Johansson2014}.

Diversas técnicas foram desenvolvidas para lidar com os problemas ocorridos em ambientes com mudanças
temporais: detecção de mudança de conceito
\cite{klinkenberg2000detecting,zenisek2019machine}, retreino periódico de algoritmo
\cite{bifet2007learning,anderson2019recurring,ramanath2020lambda} e novas abordagens de algoritmos para lidarem com esse contexto \cite{zang2014comparative,manapragada2018extremely,cano2020kappa}. Por exemplo, alguns
métodos de detecção de mudança de conceito \cite{gama2010knowledge} são
baseados em alarmes que apontam quando há diferença de desempenho do modelo
atual e do melhor modelo até o momento é maior que um dado limiar. Outros métodos,
como a indução periódica de modelos, podem não ser suficientes para resolver
esses problemas, já que o espaço de hipótese fixado do algoritmo pode não mais
ser apropriado ao problema~\cite{rossi2014metastream}.

Meta-Aprendizado (MtA) é uma técnica proeminente para resolver essas limitações
através da detecção de mudança de conceito e recomendação de algoritmos para
melhorar a predição \cite{Anderson2019,VanRijn2016,Zarmehri2015}. Uma das abordagens para uso de MtA é a construção de meta-dados \cite{Vanschoren2018,khan2020literature} baseado em uma
conjunto de características descritivas, nomeados meta-atributos, do problema
sob análise \cite{Rivolli2018,lorena2019complex}. Esses meta-atributos são extraídos e combinados
criando um meta-exemplo. Diferentes algoritmos de AM são aplicados ao problema,
e seus desempenhos são utilizados para rotular os meta-exemplos. Esse
procedimento é executado para o fluxo de dados em diferentes pontos no tempo,
resultando em um conjunto de meta-dados, o qual pode ser usado para induzir um
modelo capaz de predizer qual será o algoritmo mais apropriado para eventos
futuros sem os altos custos usuais que o processo de seleção de modelos de ML
demanda \cite{Munoz2018}.

O MetaStream, é um \textit{framework} baseado em MtA, que mantém o treinamento contínuo de dois grupos de algoritmos por janelas deslizantes, algoritmos a nível base, que vão efetivamente predizer a variável alvo do fluxo de dados, e um algoritmo a nível meta, que é treinado com as características do fluxo, descrita por meta-atributos, e os respectivos desempenhos em momentos passados dos algoritmos nível base, para então recomendar o algoritmo mais adequado a ser usado no fluxo atual \cite{rossi2012, rossi2014metastream}.

Nesse trabalho, o MetaStream é aprimorado pela extensão de meta-atributos para mais modernos e informativos \cite{Rivolli2018}, e incluindo aprendizado incremental no nível meta, propondo o  LightGBM \cite{ke2017lightgbm} como meta-classificador. Diferente do trabalho original em \cite{rossi2014metastream}, que eram usadas apenas medidas estatísticas simples como média, desvio padrão, curtose, etc. para descrever os dados, nesse trabalho estendemos por medidas estatísticas muito mais complexas, como esparsidade, Wilks' Lambda, e outros e medidas de grupos extraídos de modelos induzidos (\textit{Landmarking} e \textit{Model Based}). O LightGBM como meta-classificador fornece diversas propriedades interessantes, apresenta desempenho preditiva superior em relação a algoritmos semelhantes aliado a um menor consumo de recursos \cite{lightgbmBench}, lida nativamente com atributos faltantes, essencial para esse contexto pois em condições especificas o conjunto de meta-atributos apresenta atributos faltantes, e por fim, permite o aprendizado incremental que reduzir drasticamente o tempo de processamento e memória para induzir um novo modelo. Seguimos então para investigar se meta-atributos de ponta e aprendizado incremental provido pelo LightGBM são capazes de predizer de forma mais precisa que o método base e aprimorar o desempenho geral do sistema aprendiz.


\section{Hipóteses e Objetivos}

O problema de mudança de conceito, isto é, uma alteração na função geradora dos dados, pode afetar a performance do sistema de aprendizado em vigência significativamente, dado que o seu espaço de hipóteses pré-definido pode não aproximar tão bem essa nova distribuição dos dados. Portanto, essa constatação inicia a motivação de pesquisa e nesta seção definiremos a hipótese, objetivos e a pergunta de pesquisa que esse trabalho buscará responder.

Em fluxos de dados, ambientes dinâmicos que geram dados continuamente, e portanto, evoluem com o tempo, é comum a presença de mudança de conceito, logo, temos como \textbf{hipótese} inicial para esse trabalho que a recomendação continua de vieses distintos, apropriados as características atuais dos dados, resulta em uma sistema de aprendizado com um melhor poder preditivo em função de um único algoritmo definido \textit{apriori}.

Embora no longo prazo esses algoritmos desempenhem em média de forma similar para o fluxo de dados, faz sentido realizar a seleção dinâmica se em períodos menores do tempo, um algoritmo domina o desempenho em relação aos outros. Pois, se um viés é dominante em todo momento observado, não haverá ganhos por essa seleção. Desta forma, esse trabalho tem como \textbf{objetivo} avaliar se a recomendação de algoritmos gera ganhos preditivos ao longo do tempo.

Traçado o objetivo desse trabalho, temos então as seguintes perguntas de pesquisa:
\begin{itemize}
    \item \textbf{RQ1:} Meta-atributos mais modernos, ``estado-da-arte'' possibilitam uma recomendação do algoritmo mais adequado em um dado momento, levando a ganhos significativos de desempenho ao longo do tempo?
    \item \textbf{RQ2:} A nível meta, em que o conceito pode se manter ao longo do tempo, é possível obter poder preditivo equivalente para um algoritmo incremental, que tem menor consumo de recursos, em relação a um não-incremental?
\end{itemize}

\section{Considerações}

A partir dos estudos desse trabalho também foi concebido um artigo em conjunto com os professores Dr. André Rossi, Dr. Gustavo Batista e Dr. Luís Paulo Garcia (orientador desta monografia), que foi aceito para a conferência \textit{``25th International Conference on Pattern Recognition (ICPR), 2020''}, classificada como QUALIS A2 (2016), com título \textit{``Algorithm Recommendation for Data Streams''}, que está em anexo ao final do trabalho. 

O restante desse texto esta estruturado da seguinte forma: o Capítulo \ref{chap:aprendizado} o que já tem sido feito para resolver o problema aqui proposto e as bases teóricas das soluções aqui empregas, o Capítulo \ref{chap:metodologia} apresenta a metodologia, constituída do \textit{framework} MetaStream, modificações aplicadas a ele e as configurações do experimento, o Capítulo \ref{chap:resultados} apresenta os resultados e analises dos experimentos conduzidos, e por fim, o Capítulo \ref{chap:conclusao} conclui a monografia.